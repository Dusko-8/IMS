\section{Porovnanie Priepustnosti: Pokladna vs. Automat}
\subsection{Úvod do Experimentu}
V tomto experimente sme nahradili tradičnú pokladňu č. 2 automatizovaným platobným systémom, označeným ako `PokladnaAutomat2`, s cieľom posúdiť vplyv na priepustnosť a efektívnosť operácií v lyžiarskom stredisku.

\subsection{Výsledky Experimentu}
Porovnávame kľúčové ukazovatele výkonnosti pre  pôvodnú `Pokladna2` a novú `PokladnaAutomat2`:

\begin{table}[h!]
\centering
\begin{tabular}{@{}lcc@{}}
\toprule
Indikátor &  Pokladna2 (pôvodná)  & PokladnaAutomat2 (nová) \\ \midrule
Počet požiadaviek & 103 &  198 \\
Maximálna dĺžka radu & 17 &  17 \\
Priemerná dĺžka radu & 4 &  3 \\
Priemerná čakacia doba (minuty) & 15 & 8 \\
\bottomrule
\end{tabular}
\caption{Porovnanie výkonnosti pokladní a automatov.}
\label{tab:performance_comparison}
\end{table}

\subsection{Diskusia a Záver}
Z experimentálnych dát je zrejmé, že zavedenie `PokladnaAutomat2` viedlo k výraznému zlepšeniu. 
Priemerná čakacia doba bola o takmer polovicu nižšia v porovnaní s `Pokladna1`. 
Taktiež sme pozorovali, že maximálna a priemerná dĺžka radu zostala stabilná, napriek zvýšeniu počtu transakcií,
čo poukazuje na vyššiu efektivitu `PokladnaAutomat2`.
Na základe týchto pozorovaní odporúčame zvážiť ďalšie nahradenie tradičných
pokladní automatizovanými systémami, ktoré môžu významne zlepšiť priepustnosť a zákaznícku skúsenosť v celom stredisku. 
Automatizácia znižuje čakacie doby, zvyšuje spokojnosť zákazníkov a môže potenciálne znížiť prevádzkové náklady.

