\section{Kľúčové aspekty simulačných experimentov}

V rámci projektu kladieme dôraz na analýzu a hodnotenie priepustnosti lyžiarskeho strediska. Cieľom simulačných experimentov je 
identifikovať úzke miesta v toku návštevníkov a navrhnúť efektívne riešenia na ich odstránenie.

\subsection{Metodológia experimentov}

Experimenty budú zamerané na:
\begin{itemize}
  \item Zber a analýzu dát o pohybe návštevníkov v rôznych častiach strediska počas vybraných intervalov.
  \item Simuláciu návštevnosti a záťaže na kľúčových bodoch, ako sú vleky, reštaurácie a ďalšie služby.
  \item Hodnotenie vplyvu zmenených prevádzkových stratégií na celkovú efektivitu a spokojnosť návštevníkov.
\end{itemize}

\subsection{Očakávaný prínos experimentov}

Z experimentov očakávame získanie hĺbkového porozumenia dynamiky návštevníkov a základu pre:
\begin{itemize}
  \item Návrh zlepšení v rozložení a využití strediskových kapacít.
  \item Optimalizáciu časového rozvrhu a logistiky návštevníkov.
  \item Zlepšenie celkovej návštevníckej skúsenosti prostredníctvom zníženia čakacích dôb.
\end{itemize}

\subsection{Overenie validity modelu}
Validita modelu bola overená porovnaním výstupných dát simulácie a dát zozbieraných z lyžiarskeho strediska.
Výsledný model je schopný celkom presne reprodukovať reálnu prevádzku lyžiaskeho strediska. Ako je možné vidiet v tabulke \ref{tab:validity}, 
kde sú porovnané získané dáta a výstup zo simulácie.
\begin{table}[h!]
  \centering
  \begin{tabular}{@{}lcc@{}}
  \toprule
  Indikátor &  Reálne dáta  & Výstup simulácie \\ \midrule
  Počet návštevníkov & 1800 &  1890 \\
  Čakanie na lanovku & 1-2 min &  0,5 min \\
  Jazda lanovkou & 15 min &  15 min \\
  Zjazd svahu & 7-8 min & 8 min \\
  \bottomrule
  \end{tabular}
  \caption{Porovnanie reálnych dát a simulácie.}
  \label{tab:validity}
  \end{table}
Pozn: Čakanie na lanovku je v modeli kratšie ako v reálnom prostredí. Je to spôsobené tým že ak je v rade 4 a viac ľudí,
tak v simulácií nasadnú vždy 4 ľudia, pričom v realite tomu nie vždy tak je. Pokusy o simulovanie tohto správania a dosiahnutia
čakacej doby, ktorá je viac bližšia realite však znevalidovávali ostatné údaje simulácie.


