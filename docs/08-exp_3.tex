\section{Priepustnosť Cafeteria bar}
Ďaľším z kritických bodov lyžiarskeho strediska bol Cafeteria bar. Priemerná čakacia doba a zároveň aj maximálna čakacia doba boli príliš vysoké.
Tento experiment sa bude zaoberať znížením tejto doby. Navrhnutým riešiením je zamestnanie brigádnika, ktorý by obsluhoval druhú pokladňu v bare.
Tým pádom budeme predpokladať že mu výkon služby bude trvať dvojnásobne dlhšie ako stálemu pracovníkovi.
\subsection{Výsledky experimentu}
Porovnanie parametrov v pôvodnej konfigurácii a následne s pridaním druhej pokladne, ktorá bude obsluhovaná brigádnikom.
V tabulke \ref{tab:caf} je možné vidieť prehľad údajov pre bar s jedným zamestnancom a následne bar po pridaní brigádnika a samého birgádnika.

\begin{table}[h!]
    \centering
    \begin{tabular}{@{}lccc@{}}
    \toprule
    Indikátor &  Cafeteria bar(pôvodne)  & Cafeteria Bar(nová) & Brigadnik\\ \midrule
    Počet návštevníkov & 109 & 79 & 34\\
    Priemerné čakanie & 12 min & 4 min & 9 min\\
    Maximálny čakanie & 50 min & 16 min & 30 min\\
    \bottomrule
    \end{tabular}
    \caption{Porovnanie reálnych dát a simulácie.}
    \label{tab:caf}
    \end{table}

\subsection{Zhodnotenie experimentu}
Na základe výsledku experimentuje je možné vidieť že čakacie doby boli skrátené a viac rozložené. Zamestanie brigádnika by prinieslo väčšiu 
efektivitu v rámci špičky návštevnosti, avšak je počas dňa nebol až tak vyťažený. Z toho vyplýva prijatie tohoto zlepšenia by bolo potrebné zvážiť
aj z finančnej stránky.
